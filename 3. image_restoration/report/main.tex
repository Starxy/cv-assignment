\documentclass[UTF8]{ctexart}

\usepackage{amsmath}
\usepackage{cases}
\usepackage{cite}
\usepackage{graphicx}
\usepackage[margin=1in]{geometry}
\usepackage{fancyhdr}
\usepackage{float}
\usepackage{listings}
\usepackage{ctex}
\usepackage{xcolor}
\usepackage{fontspec}
\usepackage{titling}
\pagestyle{fancy}
\fancyhf{}
\geometry{a4paper}

\lstset{ %代码块设置
    language = Python,
    numbers=left,
    keywordstyle=\color{blue!70},
    commentstyle=\color{red!50!green!50!blue!50},
    frame=shadowbox,
    rulesepcolor=\color{red!20!green!20!blue!20},
    basicstyle=\ttfamily,
    showstringspaces=false
}

\title{Noise2Noise 图像修复实验报告}
\author{\LaTeX\ by\ 董霁兴}
\date{\today}
\pagenumbering{arabic} %设置文章页码为阿拉伯数字

\begin{document}
\fancyhf{}
\fancyhead[L]{ %页眉左侧logo
    \begin{minipage}[c]{0.9\textwidth}
        \includegraphics[height=10.5mm]{picture/logo.png}
    \end{minipage}
}
\fancyhead[C]{Noise2Noise 图像修复实验报告}
\fancyfoot[C]{\thepage}

\begin{titlepage}                                               %用于单人报告的封面 For single report 
    \centering
    \includegraphics[width=0.65\textwidth]{picture/logo_text.png}   % 插入你的图片,调整文件名和路径 Insert your picture, adjust the file name and path
    \par\vspace{1.5cm}
    {\Huge \heiti Noise2Noise 图像修复实验报告 \par} % 标题 Title
    \vspace{1cm}
    {\Large \heiti 《计算机视觉》Assignment 3\par}              % 副标题 Subtitle
    \vspace{5cm}

    % 个人信息  Personal information
    \begin{center}
        {\Large                                                 % 这里的字号也可以用别的方式修改   The font size here can also be modified in other ways
        \makebox[4em][s]{\heiti 姓名}:\underline{\makebox[15em][c]{\heiti 董霁兴}}\\
        \makebox[4em][s]{\heiti 学号}:\underline{\makebox[15em][c]{\heiti 2024317181}}\\
        \makebox[4em][s]{\heiti 班级}:\underline{\makebox[15em][c]{\heiti 计算机技术2班}}\\
        \makebox[4em][s]{\heiti 学院}:\underline{\makebox[15em][c]{\heiti 信息科学与工程学院}}\\
        }
    \end{center}

    \vfill
    \today % 日期
\end{titlepage}

\newpage

\tableofcontents  %自动根据下文创建目录

\newpage
\section{引言}

图像修复是计算机视觉领域的一项重要任务,其目标是从受损或有噪声的图像中恢复出原始图像。传统的图像修复方法通常依赖于图像的统计特性,而深度学习方法则通过学习数据中的模式来实现更为精确的修复。Noise2Noise 是一种无需干净图像作为训练目标的图像修复方法,通过学习噪声图像之间的映射关系来实现图像去噪。

本实验将通过实现 Noise2Noise 模型,深入理解其基本原理和实现方法。

\section{实验目的}
本次实验主要围绕 Noise2Noise 图像修复模型展开,通过理论学习和实践操作,达到以下目的:
\begin{enumerate}
    \item \textbf{理解 Noise2Noise 的核心设计思想}:掌握无需干净图像作为训练目标的去噪方法,理解其相比传统去噪方法的优势。
    \item \textbf{掌握 Noise2Noise 的网络结构设计}:理解基础网络在特征提取中的作用,掌握损失函数的设计原则。
    \item \textbf{实现简单的 Noise2Noise 模型}:手动实现 Noise2Noise 的核心组件,在一个较简单的数据集上训练和测试模型。
\end{enumerate}

通过完成上述目标,加深对图像修复任务的理解,掌握深度学习模型设计和实现的基本框架,为今后进一步学习更复杂的图像修复模型奠定基础。

\section{实验环境}
\subsection{硬件环境}
本实验使用个人笔记本电脑进行训练,i7 14650HX 处理器和 RTX 4060 Laptop 显卡的设备,CUDA版本为 12.6

\subsection{软件环境}
实验采用 Python 3.9,和 PyTorch 深度学习框架。自行实现了一些用于展示损失曲线的工具类。

\subsection{数据集}

本次实验使用的训练数据来自 ImageNet32x32 中的 6250 张图像。每张图片随即添加了高斯噪音,为方便加载,使用了 pytorch 的 pickle 文件格式。

\section{Noise2Noise模型设计与实现}
\subsection{整体架构}

Noise2Noise 模型采用模块化设计,主要由编码器和解码器两部分组成,并包含跳跃连接。其整体架构如下:

\subsubsection{网络结构}
模型采用模块化设计方法,主要包含以下组件:
\begin{enumerate}
    \item \textbf{基础卷积块 (ConvLeakyReLU)}:
    \begin{itemize}
        \item 卷积层
        \item LeakyReLU 激活函数
    \end{itemize}
    \item \textbf{编码器模块}:
    \begin{itemize}
        \item 多个卷积块串联
        \item 最大池化层用于下采样
        \item 跳跃连接的特征保存
    \end{itemize}
    \item \textbf{解码器模块}:
    \begin{itemize}
        \item 上采样层
        \item 与编码器对应层的跳跃连接
        \item 特征融合与重建
    \end{itemize}
\end{enumerate}

\subsubsection{跳跃连接设计}
采用堆栈系统实现跳跃连接:
\begin{itemize}
    \item 编码阶段将特征图压入堆栈
    \item 解码阶段依次弹出特征图进行特征融合
    \item 确保特征的有效传递和重建
\end{itemize}

\subsection{训练策略}
\subsubsection{数据预处理}
\begin{itemize}
    \item 输入图像归一化到 [0,1] 范围
    \item 采用自定义数据集类进行批处理
    \item 实现数据增强策略
\end{itemize}

\subsubsection{超参数调优}
为了获得更好的 PSNR 分数,我们尝试了不同的修改和超参数调整,如不同的激活函数、引入批量归一化和优化器的学习率。

我们使用 Ray和 HyperOpt 库在搜索空间上进行并行优化。结果如表\ref{tab:hyperparameters}所示。我们得出结论:ReLU 激活并不比 LeakyReLU 好;使用批量归一化使训练速度慢1.5倍,同时结果略差;高学习率和小动量的效果与小学习率和大动量的效果相当。


\begin{table}[htbp]
    \centering
    \caption{不同参数配置下的模型性能对比}
    \begin{tabular}{|c|c|c|c|c|c|}
    \hline
    损失 & 激活函数 & BN & 学习率 & 动量 & 时间(秒) \\
    \hline
    0.00375 & LeakyRelu & FALSE & 0.05521 & 0.18 & 3780 \\
    0.00378 & ReLu & FALSE & 0.00801 & 0.86 & 3796 \\
    0.00383 & LeakyRelu & TRUE & 0.03571 & 0.18 & 5306 \\
    0.00448 & ReLu & FALSE & 0.00130 & 0.24 & 2612 \\
    0.00458 & ReLu & FALSE & 0.00081 & 0.43 & 3774 \\
    0.00462 & LeakyRelu & FALSE & 0.00059 & 0.44 & 3779 \\
    0.00480 & LeakyRelu & TRUE & 0.00178 & 0.21 & 5345 \\
    0.00480 & ReLu & TRUE & 0.00404 & 0.31 & 5031 \\
    0.00494 & LeakyRelu & FALSE & 0.00018 & 0.66 & 3274 \\
    0.00605 & ReLu & TRUE & 0.00024 & 0.81 & 5227 \\
    \hline
    \end{tabular}
    \label{tab:hyperparameters}
    \end{table}

\subsubsection{数据增强方法}
实验了多种数据增强策略:
\begin{itemize}
    \item 水平和垂直翻转
    \item 源图像和目标图像随机交换
    \item 色调和亮度调整(效果不理想,未采用)
\end{itemize}

\section{实验结果分析}

\subsection{训练过程分析}
\begin{figure}[htbp]
    \centering
    \includegraphics[width=0.8\textwidth]{picture/noise2noise_training_curve.png}
    \caption{Noise2Noise模型训练过程中的损失变化曲线}
    \label{fig:training_curve}
\end{figure}

\begin{itemize}
    \item \textbf{损失变化}:
    \begin{itemize}
        \item 在训练初期损失快速下降,表明模型开始学习噪声图像之间的映射关系
        \item 之后进入平稳期,损失缓慢下降
        \item 最终趋于稳定,表明模型去噪能力得到了良好训练
    \end{itemize}
\end{itemize}

\subsection{训练效率分析}
\begin{itemize}
    \item \textbf{硬件环境}:RTX 4060 Laptop
    \item \textbf{单轮训练时间}:约 60 秒/epoch
    \item \textbf{最终 PSNR}:达到 25.51dB
\end{itemize}

\subsection{模型推理效果展示}

\begin{figure}[htbp]
    \centering
    \includegraphics[width=0.9\textwidth]{picture/noise2noise_results.jpg}
    \caption{Noise2Noise模型在不同噪声类型下的推理结果}
    \label{fig:detection_results}
\end{figure}

从图\ref{fig:detection_results}的检测结果可以观察到以下特点:

\begin{itemize}
    \item \textbf{去噪成功案例}:
    \begin{itemize}
        \item 模型在大多数图像中成功去除了噪声,恢复了图像的细节
    \end{itemize}
    
    \item \textbf{存在的问题}:
    \begin{itemize}
        \item 在某些高强度噪声下,去噪效果不理想
    \end{itemize}
\end{itemize}

\section{实验结论}

本次实验通过对 Noise2Noise 模型的实现和测试,验证了其在无需干净图像作为训练目标的情况下,能够有效去除图像噪声。实验结果表明,Noise2Noise 在大多数情况下能够恢复图像的细节,但在高强度噪声下仍有改进空间。

\section{总结与思考}

\subsection{Noise2Noise的特点}

\subsubsection{模型结构特点}
\begin{itemize}
    \item \textbf{无需干净图像}:通过学习噪声图像之间的映射关系实现去噪
    \item \textbf{轻量级设计}:简化的网络结构,计算效率高
\end{itemize}

\subsubsection{优势与局限性}
\begin{itemize}
    \item \textbf{优势}:
    \begin{itemize}
        \item 适用于无干净图像的场景
        \item 计算效率高,适合资源受限场景
    \end{itemize}
    
    \item \textbf{局限性}:
    \begin{itemize}
        \item 对高强度噪声的去噪效果有限
    \end{itemize}
\end{itemize}

\subsection{改进方向}

\subsubsection{可能的优化方向}
\begin{enumerate}
    \item \textbf{网络结构优化}:
    \begin{itemize}
        \item 引入更强大的特征提取骨干网络
        \item 添加注意力机制,提升关键特征的提取能力
    \end{itemize}
    
    \item \textbf{训练策略改进}:
    \begin{itemize}
        \item 采用更先进的数据增强技术
        \item 改进损失函数设计,平衡去噪和细节保留
    \end{itemize}
\end{enumerate}

\end{document}